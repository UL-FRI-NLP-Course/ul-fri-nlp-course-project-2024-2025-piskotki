%%%%%%%%%%%%%%%%%%%%%%%%%%%%%%%%%%%%%%%%%
% FRI Data Science_report LaTeX Template
% Version 1.0 (28/1/2020)
% 
% Jure Demšar (jure.demsar@fri.uni-lj.si)
%
% Based on MicromouseSymp article template by:
% Mathias Legrand (legrand.mathias@gmail.com) 
% With extensive modifications by:
% Antonio Valente (antonio.luis.valente@gmail.com)
%
% License:
% CC BY-NC-SA 3.0 (http://creativecommons.org/licenses/by-nc-sa/3.0/)
%
%%%%%%%%%%%%%%%%%%%%%%%%%%%%%%%%%%%%%%%%%


%----------------------------------------------------------------------------------------
%	PACKAGES AND OTHER DOCUMENT CONFIGURATIONS
%----------------------------------------------------------------------------------------
\documentclass[fleqn,moreauthors,10pt]{ds_report}
\usepackage[english]{babel}

\graphicspath{{fig/}}




%----------------------------------------------------------------------------------------
%	ARTICLE INFORMATION
%----------------------------------------------------------------------------------------

% Header
\JournalInfo{FRI Natural language processing course 2025}

% Interim or final report
\Archive{Project report} 
%\Archive{Final report} 

% Article title
\PaperTitle{Conversational Agent with Retrieval-Augmented Generation} 

% Authors (student competitors) and their info
\Authors{Blaž Grilj, Ana Poklukar, Kristjan Sever}

% Advisors
\affiliation{\textit{Advisors: Aleš Žagar}}

% Keywords
\Keywords{NLP, LLM, RAG}
\newcommand{\keywordname}{Keywords}


%----------------------------------------------------------------------------------------
%	ABSTRACT
%----------------------------------------------------------------------------------------

\Abstract{
We propose the development of a conversational agent that utilizes Retrieval-Augmented Generation (RAG) to improve the accuracy and relevance of responses by grounding them in real-time external knowledge. Unlike traditional large language models (LLMs) that rely solely on pre-trained data, our envisioned system will retrieve information—primarily from Wikipedia—to enhance its ability to answer general knowledge and current events queries. The project will explore different approaches to model architecture, including the use of open-source LLMs and fine-tuning smaller pre-trained models. Key components will include a Wikipedia-focused retrieval system, a query processing pipeline, and context integration for response generation. To evaluate our approach, we plan to benchmark the RAG-enhanced system against a non-augmented baseline using metrics such as factual accuracy, hallucination rate, and response relevance. This proposal outlines the initial design and objectives of a system aimed at demonstrating the potential of RAG in conversational AI.
}

%----------------------------------------------------------------------------------------

\begin{document}

% Makes all text pages the same height
\flushbottom 

% Print the title and abstract box
\maketitle 

% Removes page numbering from the first page
\thispagestyle{empty} 

%----------------------------------------------------------------------------------------
%	ARTICLE CONTENTS
%----------------------------------------------------------------------------------------

\section*{Introduction}

The development of large language models (LLMs) has significantly advanced the capabilities of conversational agents. However, traditional LLM-based chatbots rely solely on pre-trained knowledge, which limits their ability to provide accurate and up-to-date information, especially in dynamic or domain-specific contexts. Retrieval-Augmented Generation (RAG) offers a promising solution by combining real-time information retrieval with generative language models, allowing agents to ground their responses in relevant external data.

In this project, we’re building a conversational agent that uses Retrieval-Augmented Generation (RAG) to pull in information from the web in real time. The focus is on answering general knowledge questions—anything from historical facts to current events—with responses that are accurate, coherent, and relevant. Instead of relying only on what it learned during training, the agent will actively search for up-to-date information and use it to generate better answers.
%------------------------------------------------

\section*{Related Work}

Retrieval-Augmented Generation has emerged as a promising approach to enhance the performance of large language models by addressing key limitations such as factual hallucinations, outdated knowledge, and the lack of domain-specific expertise. While LLMs like ChatGPT (OpenAI, 2022) have demonstrated impressive general capabilities (Bang et al. \cite{bang}), they are still prone to producing incorrect or outdated information (Cao et al. \cite{cao}). These shortcomings are particularly problematic in scenarios requiring factual accuracy and current information.

To address these issues, RAG integrates external information sources—often retrieved via search engines or document databases—into the response generation process. By grounding responses in real-time data, RAG significantly improves the factual consistency and relevance of generated outputs (Guu et al. \cite{guu}). This method is particularly effective when paired with robust retrieval systems, which can provide up-to-date context from the vast and constantly evolving content available online.

The field of RAG is evolving rapidly. Gao et al. \cite{gao} present a comprehensive survey that categorizes over 100 RAG-related studies into three core research paradigms and outlines the technical challenges across the stages of retrieval, generation, and augmentation. Their work offers a detailed roadmap of the current landscape and future directions of RAG in the context of LLMs.

Complementing this, Chen et al. \cite{chen} propose the Retrieval-Augmented Generation Benchmark (RGB), the first benchmark designed to systematically evaluate four core capabilities of RAG-enabled LLMs in both English and Chinese. Their analysis highlights several current limitations of LLMs in RAG settings, including difficulties in accurate retrieval, context integration, and factually consistent generation. The study also suggests concrete directions for future improvement, making it a key contribution to the evaluation of RAG systems.

%------------------------------------------------

\section*{Initial Idea}

Our plan is to create a conversational agent using Naive RAG implementation \cite{gao} for a general knowledge LLM with Wikipedia and/or some news outlet as the primary external information source. Since most of the pretrained models like LLama and DeepSeek have been trained on Wikipedia already, the best way to implement RAG for it would be to try to supplement the queries with data after the cut-off date, before which the model was trained.

\subsection*{Model Foundation}
Our work will utilize a Large Language Model (LLM) as the foundation for the RAG system. We have two potential approaches:

\begin{itemize}[itemsep=0em]
\item Leverage an existing open-source LLM such as LLAMA or similar models, which would reduce computational requirements but may limit customization
\item Train a smaller custom model with faculty's high performance compute.
\end{itemize}
Given resource constraints, we anticipate using a pre-trained model like LLAMA that can be efficiently fine-tuned or adapted to our specific use case without extensive retraining.

\subsection*{Information Retrieval Component}
The system will incorporate a web scraping module focused primarily on Wikipedia as an information source due to its breadth of knowledge and structural consistency. This will involve:

\begin{itemize}[itemsep=0em]
\item Developing a robust web scraper capable of efficiently retrieving Wikipedia articles
\item Implementing rate limiting and caching strategies to respect website policies and reduce redundant requests
\item Creating parsers to extract relevant content while preserving semantic structure
\end{itemize}

\subsection*{Query Processing System}
A critical challenge will be developing an effective query processing system that can:

\begin{itemize}[itemsep=0em]
\item Extract key information needs from potentially ambiguous user prompts
\item Transform these needs into specific search queries optimized for Wikipedia's search system
\item Handle disambiguation when multiple possible interpretations exist
\item Dynamically refine queries based on initial search results
\item Provide context from retrieved documents in a LLM frendly way, so not including too much information and ranking it based on relevance.
\end{itemize}

\subsection*{Evaluation Framework}
To measure the effectiveness of our RAG implementation, we will establish a benchmark comparing:

\begin{itemize}[itemsep=0em]
\item Performance of the base LLM without retrieval augmentation
\item The same model enhanced with our RAG system
\item Metrics will include factual accuracy, hallucination reduction rate, response relevance, and response completeness
\item Use external benchmarking and ideas from \cite{chen} 
\end{itemize}

This comparative analysis will provide quantitative evidence of how external knowledge retrieval impacts model performance across different query types and knowledge domains.

\subsection*{Datasets, data and corpus}
To support our project, we will use a variety of datasets and corpora. Our primary corpus will be Wikipedia Dump, which provides a broad spectrum of general knowledge. In addition, we will utilize Wikidata to complement Wikipedia dump, offering structured knowledge and detailed entity relationships for enhanced context. To ensure our data remains up-to-date, we will continuously scrape Wikipedia pages for the latest updates. For evaluation purposes, we plan to use datasets like Natural Questions\footnote{ai.google.com/research/NaturalQuestions} to benchmark retrieval and generation performance, along with TriviaQA\footnote{nlp.cs.washington.edu/triviaqa} to assess factual accuracy.  



%----------------------------------------------------------------------------------------
%	REFERENCE LIST
%----------------------------------------------------------------------------------------
\bibliographystyle{unsrt}
\bibliography{report}


\end{document}